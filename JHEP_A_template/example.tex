
\documentclass[11pt]{article}
\usepackage{jheppub}

\usepackage{amssymb}
\usepackage{amsmath}
%\usepackage{epsfig}

%\usepackage{graphicx, subfigure, array, placeins, float}



%%%%%%%%%%%%%%%%%%%%%%%%%%%%%%%%%%%%%%%
%%%%%%%%%%%
%%   todo notes   %%
%%%%%%%%%%%
\usepackage[textsize=footnotesize,textwidth=2.25cm]{todonotes}

\newcommand{\lin}[2]{\textcolor{blue}{#1}\todo[color=yellow]{J: #2}}
\newcommand{\yang}[2]{\textcolor{blue}{#1}\todo[color=orange]{Y: #2}}

%\lin{}{comments here}
%\yang{}{comments here.}

%%%%%%%%%%%%%%%%%%%%%%%%%%%%%%%%%%%%%%%



\title{Title here}



\author[a]{Guanda Lin}
\emailAdd{pleaseguess@pku.edu.cn}
\affiliation[a]{School of Physics, PKU}

\abstract{
Abstract.
}


%%%%%%%%%%%%%%%%%%%%
%%%%%%%%%%%%%%%%%%%%


\begin{document}


\maketitle


\setcounter{footnote}{0}


%%%%%%%%%%%%%%%%%%%%%%%%%%%%%%%
%%%%%%%%%%%%%%%%%%%%%%%%%%%%%%%
\section{Introduction}

%{\bf [1. Background]}

%%%%%%%%%%%%%%%%%%%%%%%%%%%%%%
%\begin{figure}[tb]
%\centering
%\includegraphics[scale=0.28]{FF_3g_2loop_MIs.eps}
%\caption{The master integrals of the two-loop 3-point form factor.}
%\label{fig:FF_3g_2loop_MIs}
%\end{figure}
%%%%%%%%%%%%%%%%%%%%%%%%%%%%%%


%\newpage
%%%%%%%%%%%%%%%%%%%%%%%%%%%%%%%


%%%%%%%%%%%%%%%%%%%%%%%%%%%%%%%
%%%%%%%%%%%%%%%%%%%%%%%%%%%%%%%
\section{Discussion}
\label{sec:discussion}



%%%%%%%%%%%%%%%%%%%%%%%%%%%%%%%%
\acknowledgments

It is a pleasure to thank XXX for discussions. 
This work is supported in part by the National Natural Science Foundation of China (Grants No. 11822508, 11847612),
by the Chinese Academy of Sciences (CAS) Hundred-Talent Program, 
and by the Key Research Program of Frontier Sciences of CAS. 



\appendix


\section{One-loop results}
\label{app:oneloop}



\bibliographystyle{jhep}

\bibliography{reference}


\end{document}




